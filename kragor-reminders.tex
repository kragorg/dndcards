\documentclass[landscape]{article}
\usepackage[paperwidth=6in, paperheight=4in, margin=0.25in]{geometry}
\usepackage{tikz}
\usetikzlibrary{calc}
\usepackage{array}
\usepackage{booktabs}
\usepackage{keyval}
\usepackage[skip=0em]{parskip}
\usepackage{xcolor}
\usepackage{ETbb}

\renewcommand{\rmdefault}{etbb}
\renewcommand{\sfdefault}{etbb}
\definecolor{headercolor}{HTML}{8B2020}

\newcommand{\CardTitle}[2][]{%
  {%
    \Large\textbf{\textcolor{headercolor}{\textsc{#2}}}%
    \hfill%
    \ifx&#1&%
    \else%
      {\small\textit{#1}}%
    \fi%
  }

  {\rule{\linewidth}{0.4pt}}%
}

\newenvironment{Card}[2][]{%
  \CardTitle[#1]{#2}%
  \setlength{\parskip}{0.1in}%
  }{\newpage}

\newcommand{\head}[1]{\textbf{\small \textsc{\textcolor{headercolor}{#1}}}}
\newcommand{\m}[2][]{%
  \textit{#2%
  \ifx\relax#1\relax%
  \else{ (\uppercase\expandafter{\romannumeral #1})}%
  \fi}}
\newcommand{\smol}[1]{\footnotesize #1}
\newcommand{\freq}[1]{\smol{(#1)}}

\newcommand{\HPACInitiative}[3]{
  \begin{center}
  \begin{tikzpicture}
  \node[draw=headercolor, line width=0.5pt, minimum width=1.2cm, minimum height=1.2cm, align=center] (hp) {
    {\large \textbf{#1}}\\[0.1cm]
    {\footnotesize \textbf{HP}}
  };
  \node[draw=headercolor, line width=0.5pt, minimum width=1.2cm, minimum height=1.2cm, align=center] (ac) at ($(hp.east)+(1cm,0)$) {
    {\large \textbf{#2}}\\[0.1cm]
    {\footnotesize \textbf{AC}}
  };
  \node[draw=headercolor, line width=0.5pt, minimum width=1.2cm, minimum height=1.2cm, align=center] (init) at ($(ac.east)+(1cm,0)$) {
    {\large \textbf{#3}}\\[0.1cm]
    {\footnotesize \textbf{Initiative}}
  };
  \end{tikzpicture}
  \end{center}
}

\makeatletter
\newcommand{\skill}{\@ifstar{\@skillp}{\@skill}}
\newcommand{\@skill}[2]{%
  \footnotesize {\bfseries #1} \hfill #2}
\newcommand{\@skillp}[2]{%
  \footnotesize {\bfseries \itshape \scshape #1} \hfill #2}

\define@key{spell@keys}{tags}{\def\spell@tags{#1}}%
\define@key{spell@keys}{level}{\def\spell@level{#1}}%
\define@key{spell@keys}{time}{\def\spell@time{#1}}%
\define@key{spell@keys}{range}{\def\spell@range{#1}}%
\define@key{spell@keys}{components}{\def\spell@components{#1}}%
\define@key{spell@keys}{duration}{\def\spell@duration{#1}}%
\define@key{spell@keys}{school}{\def\spell@school{#1}}%
\define@key{spell@keys}{attack}{\def\spell@attack{#1}}%
\define@key{spell@keys}{effect}{\def\spell@effect{#1}}%
\setkeys{spell@keys}{tags={}}%

\NewDocumentEnvironment{Spell}{om}{%
  \setkeys{spell@keys}{#1}%
  \CardTitle[\spell@tags]{#2}
  \begin{tabular}{p{0.22\linewidth}p{0.22\linewidth}p{0.22\linewidth}p{0.22\linewidth}}
  \head{Level}            & \head{Casting Time}   & \head{Range/Area}     & \head{Components}         \\
  \smol{\spell@level}     & \smol{\spell@time}    & \smol{\spell@range}   & \smol{\spell@components}  \\
  \head{Duration}         & \head{School}         & \head{Attack/Save}    & \head{Damage/Effect}      \\
  \smol{\spell@duration}  & \smol{\spell@school}  & \smol{\spell@attack}  & \smol{\spell@effect}
  \end{tabular}

  {\rule{\linewidth}{0.4pt}}
  \setlength{\parskip}{0.1in}
}{\newpage}

\NewDocumentEnvironment{components}{}{%
  \vfill
  \begin{tabular}{rl}
}{\end{tabular}}

\makeatother


\begin{document}
\pagestyle{empty}

\CardTitle{Kragor}

\begin{tabular}{p{0.3\linewidth}p{0.3\linewidth}p{0.1\linewidth}p{0.3\linewidth}}%
  \head{Attack}            & \head{Range}     & \head{Hit}     & \head{Damage} \\
  Conjured Warhammer       & 5 feet (melee)   & +5             & 1d10+3 Psychic \\
  \m{Eldritch Blast}       & 120 feet         & +5             & 1d10+3 Force \\
\end{tabular}

{\rule{\linewidth}{0.4pt}}
\vspace{1em}

\begin{tabular}{p{0.24\linewidth}p{0.24\linewidth}p{0.48\linewidth}}
  \multicolumn{2}{l}{\head{Actions}}                               & \head{Bonus Actions} \\
  Dash                     & \m{Mage Hand}                         & \m[1]{Armor of Agathys} \\
  \m[1]{Detect Magic}      & Magic                                 & Adrenaline Rush \freq{2/short rest} \\
  Disengage                & Ready                                 & \m[1]{Hex} \\
  Dodge                    & Search                                & Pact of the Blade: Bond \\
  \m{Fiendish Vigor}       & Shove                                 & Pact of the Blade: Conjure \\
  Grapple                  & Study                                 & \\
  Help                     & Utilize                               & \head{Reactions} \\
  Hide                     &                                       & Opportunity Attack \\
  Improvise                &                                       & \\
  Influence                &                                       & \head{Other} \\
                           &                                       & Luck \freq{2/long rest} \\
                           &                                       & Magical Cunning \freq{1/long rest} \\
                           &                                       & Relentless Endurance \freq{1/long rest}
\end{tabular}


\newpage
\CardTitle{Kragor}

\begin{tabular}{r@{\hspace{0.5em}}c@{\hspace{0.5em}}c@{\hspace{0.5em}}c@{\hspace{0.5em}}c@{\hspace{0.5em}}c@{\hspace{0.5em}}c}
                  & \head{Strength} & \head{Dexterity} & \head{Constitution} & \head{Intelligence} & \head{Wisdom} & \head{Charisma} \\
\textbf{Score}    & 10              & 14               & 15                  & 8                   & 10            & 17 \\
\textbf{Modifier} & \smol{+0}       & \smol{+2}        & \smol{+2}           & \smol{-1}           & \smol{+0}     & \smol{+3} \\
\textbf{Save}     & \smol{+0}       & \smol{+2}        & \smol{+2}           & \smol{-1}           & \smol{+2}     & \smol{+5} \\
\end{tabular}

{\rule{\linewidth}{0.4pt}}

\HPACInitiative{17}{13}{+2}

\begin{center}
  \head{Skills}
\end{center}

\begin{center}
\begin{tabular}{p{0.26\linewidth}@{\hspace{0.8cm}}p{0.26\linewidth}@{\hspace{0.8cm}}p{0.26\linewidth}}
  \skill{Acrobatics}{+2}       & \skill{History}{-1}        & \skill{Performance}{+3} \\
  \skill{Animal Handling}{+0}  & \skill*{Insight}{+2}       & \skill{Persuasion}{+3} \\
  \skill*{Arcana}{+1}          & \skill*{Intimidation}{+5}  & \skill{Religion}{-1} \\
  \skill{Athletics}{+0}        & \skill{Investigation}{-1}  & \skill{Sleight of Hand}{+2} \\
  \skill{Deception}{+3}        & \skill{Medicine}{+0}       & \skill*{Stealth}{+4} \\
  \skill{Nature}{-1}           & \skill{Perception}{+0}     & \skill{Survival}{+0} \\
\end{tabular}
\end{center}

\newpage


\begin{Card}{Magical Cunning}
You can perform an esoteric rite for 1 minute. At the end of it, you regain expended Pact Magic spell slots but no more than a number equal to half your maximum (round up). Once you use this feature, you can't do so again until you finish a Long Rest.
\end{Card}


\begin{Card}{Pact of the Blade}
As a Bonus Action, you can conjure a pact weapon in your hand—a Simple or Martial Melee weapon of your choice with which you bond—or create a bond with a magic weapon you touch; you can't bond with a magic weapon if someone else is attuned to it or another Warlock is bonded with it. Until the bond ends, you have proficiency with the weapon, and you can use it as a Spellcasting Focus. \textit{“Malleum Evoco” (MAHL-leh-um eh-VOH-coh)}.

Whenever you attack with the bonded weapon, you can use your Charisma modifier for the attack and damage rolls instead of using Strength or Dexterity; and you can cause the weapon to deal Necrotic, Psychic, or Radiant damage or its normal damage type.

Your bond with the weapon ends if you use this feature's Bonus Action again, if the weapon is more than 5 feet away from you for 1 minute or more, or if you die. A conjured weapon disappears when the bond ends.
\end{Card}


\begin{Card}[Eldritch Blast]{Agonizing Blast}
You can add your Charisma modifier to the Eldritch Blast spell's damage rolls.
\end{Card}


\begin{Card}{Fiendish Vigor}
You can cast \textit{False Life} on yourself without expending a spell slot. When you cast the spell with this feature, you don't roll the die for the Temporary Hit Points; you automatically get the highest number on the die.

\textbf{False Life}: You gain 2d4 + 4 Temporary Hit Points.
\end{Card}


\begin{Card}{Adrenaline Rush}
As a Bonus Action, you can take the Dash action. When you do, you gain 2 Temporary HP. You can use this trait 2 times, and you regain all uses after a Short or Long Rest.
\end{Card}


\begin{Card}{Relentless Endurance}
Once per Long Rest, when you are reduced to 0 HP but not killed outright, you can drop to 1 HP instead.
\end{Card}


\begin{Card}{Darkvision \textit{(120 ft)}}
You have Darkvision with a range of 120 ft.
\end{Card}


\begin{Card}{Lucky}
\textbf{Luck Points.} You have a number of Luck Points equal to your Proficiency Bonus and can spend the points on the benefits below. You regain your expended Luck Points when you finish a Long Rest.

\textbf{Advantage.} When you roll a d20 for a D20 Test, you can spend 1 Luck Point to give yourself Advantage on the roll.

\textbf{Disadvantage.} When a creature rolls a d20 for an attack roll against you, you can spend 1 Luck Point to impose Disadvantage on that roll.
\end{Card}

\begin{Card}{Thieves' Tools}
Perhaps the most common tools used by adventurers, thieves’ tools are designed for picking locks and foiling traps. Proficiency with the tools also grants you a general knowledge of traps and locks.

\textbf{Components.} Thieves’ tools include a small file, a set of lock picks, a small mirror mounted on a metal handle, a set of narrow-bladed scissors, and a pair of pliers.

\textbf{History.} Your knowledge of traps grants you insight when answering questions about locations that are renowned for their traps.

\textbf{Investigation and Perception.} You gain additional insight when looking for traps, because you have learned a variety of common signs that betray their presence.

\textbf{Set a Trap.} Just as you can disable traps, you can also set them. As part of a short rest, you can create a trap using items you have on hand. The total of your check becomes the DC for someone else’s attempt to discover or disable the trap. The trap deals damage appropriate to the materials used in crafting it (such as poison or a weapon) or damage equal to half the total of your check, whichever the DM deems appropriate.

\begin{tabular}{lc}
\textbf{Activity} & \textbf{DC} \\
Pick a lock & Varies \\
Disable a trap & Varies \\
\end{tabular}
\end{Card}

\end{document}
